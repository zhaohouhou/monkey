% ----------------------------------------------------------------
% Article Class (This is a LaTeX2e document)  ********************
% ----------------------------------------------------------------
\documentclass{article}
\usepackage{geometry}  %ZD: text padding
\usepackage[english]{babel}
\usepackage{styfiles/proof, styfiles/code, amsmath,amsthm}
\usepackage{amssymb}  % for "\leadsto"
\usepackage{bbold}%for typeface: mathbb
\usepackage{hyperref}
% THEOREMS -------------------------------------------------------
\newtheorem{thm}{Theorem}[section]
\newtheorem{cor}[thm]{Corollary}
\newtheorem{lem}[thm]{Lemma}
\newtheorem{prop}[thm]{Proposition}
\theoremstyle{definition}
\newtheorem{defn}[thm]{Definition}
\theoremstyle{remark}
\newtheorem{rem}[thm]{Remark}
\numberwithin{equation}{section}

% ----------------------------------------------------------------
\begin{document}

\newcommand{\env}[1]{[\![#1]\!]\kappa}
\newcommand{\round}[1]{(\!|#1|\!)}

\title{Defunctionalization}%
\author{Di Zhao}%
%\address{address}%
%\thanks{}%\sqrt{}
\date{\small{\texttt{zhaodi01@mail.ustc.edu.cn}}}%
% ----------------------------------------------------------------

\maketitle
% ----------------------------------------------------------------

Document for the defunctionalization phase.

This stage will eliminate the
lambda-absraction. The input is the CPS representaion, the output of this
stage is the FLAT representation.

\section{Calculating free variables}

Still, the first step is to the free variable calculation.
The input in this phase is the CPS representation (Figure 1).

\begin{figure}[!ht]
  % Requires \usepackage{graphicx}
  \centering
\begin{tabular}{rrcl}
(terms) & $K$ & $\to$ & \textsf{letval }$x\ =\ V$ \textsf{ in } $K$ \\
        &     & $|$ & \textsf{letcont }$k\ x = K$\textsf{ in }$K'$\\
        &     & $|$ &  $k\ x$ \\
        &     & $|$ & $f\ k\ x$ \\
        &     & $|$ & \textsf{case} $x$ \textsf{of}
            $\overrightarrow{\texttt{in}i_j\ x_j\ \texttt{=>}\ K_j}$\\
        &     & $|$ & \textsf{letprim} $x=PrimOp\ \vec{y}
         \texttt{ in}\ K$\\
        &     & $|$ &\textsf{if} $x$ \textsf{then} $k_1\ \textsf{else}\ k_2$\\
        &     & $|$ &\textsf{letfix }$f\ k\ x=K$\textsf{ in }$K'$\\

(values) & $V$ & $\to$ & () $\ \ |$ \textsf{ true } $|$ \textsf{ false}\\
        &     & $|$ & $i\ \ $ $\ |\ $  $\ "s"$\\
        &     & $|$ & ($x_1,x_2, ..., x_n$)\\
        &     & $|$ & \textsf{in}$i\ x$\\
        &     & $|$ &  $\lambda k\ x.K$ \\
        &     & $|$ &  $\texttt{\#}i\ x$ \\

(primitive & $PrimOp$ & $\to$ & $+\ $ $|\ -\ |$ $\ *$\\
operations) &     & $|$ & $>\ |\ <\ |\ =$\\
        &     & $|$ & $\textsf{andalso }\ |\ \textsf{orelse }\ |\ \textsf{ not}$\\
        &     & $|$ & \textsf{print} $|$ \textsf{int2string}\\
\end{tabular}
  \caption{CPS syntax}
  \label{fig-sub}
\end{figure}

The procedure is identical will that in the closure conversion
. The functions are illustrated in Figure 2 and Figure 3.

 (The code should be reusable. However as we put the free variable information in the
  closure syntax tree earlier, so the function code cannot be reused directly. Maybe its
  better to store that in the CPS syntax tree.)

\begin{figure}[!ht]
  % Requires \usepackage{graphicx}
  \centering
\begin{align*}
\mathcal{F}\ &:\ \textrm{Cps.t} \to \textrm{string set}\\     %head
\mathcal{F}(\textsf{letval }x=V\textsf{ in }K)
    &=(\mathcal{F}(K)/x)\cup \mathcal{H}(V)\\   %letval
\mathcal{F}(\textsf{letcont }k\ x=K\textsf{ in }K')
    &=(\mathcal{F}(K)/x)\cup (\mathcal{F}(K')/k)\\   %letcont
\mathcal{F}(k\ x)
    &=\{k, x\}\\   %cont apply
\mathcal{F}(f\ k\ x)
    &=\{f, k, x\}\\   %func apply
\mathcal{F}(\textsf{case}\ x\ \textsf{of}
            \ \overrightarrow{\texttt{in}i_j\ x_j\ \texttt{=>}\ K_j})
    &=\{x\}\cup (\mathcal{F}(K_1)/x_1)\cup ... \cup (\mathcal{F}(K_n)/x_n)\\
        & (\textrm{where}\ j = 1,\ ...\ , n)\\   %case
\mathcal{F}(\textsf{letprim }x=PrimOp\ \vec{y}\textsf{ in}\ K)
    &=(\mathcal{F}(K)/x)\cup \texttt{set}(\vec{y})\\   %let prims
\mathcal{F}(\textsf{if }x\textsf{ then }k_1\ \textsf{else }k_2)
    &=\{x, k_1, k_2\}\\   %if
\mathcal{F}(\textsf{letfix }f\ k\ x=K\textsf{ in }K')
    &=(\mathcal{F}(K)-\{f,k,x\})\cup (\mathcal{F}(K')/f)   %let fix
\end{align*}
  \caption{Calculating Free Variables in CPS terms}
  \label{fig-sub}
\end{figure}

\begin{figure}[!ht]
  % Requires \usepackage{graphicx}
  \centering
\begin{align*}
\mathcal{H}\ &:\ \textrm{Cps.v} \to \textrm{string set}\\     %head
\mathcal{H}(()) &={\O}\\   %empty
\mathcal{H}(\texttt{true}) &={\O}\\   %true
\mathcal{H}(\texttt{false}) &={\O}\\   %false
\mathcal{H}(i) &={\O}\\   %integer
\mathcal{H}("s") &={\O}\\   %string
\mathcal{H}((x_1, x_2,\ ...\ , x_n)) &=\{x_1, x_2,\ ...\ , x_n\}\\   %tuple
\mathcal{H}(\textsf{in}i\ x) &= \{x\}\\        %tag
\mathcal{H}(\texttt{\#}i\ x) &= \{x\}\\        %proj
\mathcal{H}(\lambda k\ x.K) &= \mathcal{F}(K)-\{k, x\}      %abstraction
\end{align*}
  \caption{Calculating Free Variables in CPS values}
  \label{fig-sub}
\end{figure}


\section{Target Syntax}

The final output of this stage is a defunctionalized syntax.
We name it as Defunc. The syntax is shown in Figure 4.

A program consists of two functions: $applycont$, $applyfunc$ and an expression.
The function $applyfunc$ takes three arguments $(f, k, x)$ and its body is a case
expression whose condition is $f$. Similarly, The function $applycont$ takes two
arguments $(k, x)$ and its body is a case expression whose condition is $k$.

As for terms (expressions), there won't be any abstractions or function definitions.
Function definitions are replaced by tagged values and the code will be inserted into
the $applycont$ or $applyfunc$ function.

\begin{figure}[!ht]
  % Requires \usepackage{graphicx}
  \centering
\begin{tabular}{rrcl}
(program) & $p$ & $\to$ & $applyfunc(f, k, x)\{K_1\}$\\
        &       &       & $applycont(k_1, x_1)\{K_2\}$\\
        &       &       &  $K_3$\\

(terms) & $K, T$ & $\to$ & \textsf{letval }$x\ =\ V$ \textsf{ in } $K$ \\
        &     & $|$ &  $applycont(k,\ x)$ \\
        &     & $|$ &  $applyfunc(f,\ k,\ x)$ \\
        &     & $|$ & \textsf{case} $x$ \textsf{of}
             $\overrightarrow{\textsf{in}i_j\ x_j \Rightarrow K_j}$\\
        &     & $|$ & \textsf{letprim} $x=PrimOp\ \vec{y}
         \textsf{ in}\ K$\\
        &     & $|$ &\textsf{if} $x$ \textsf{then} $K\ \textsf{else}\ K'$\\

(values) & $V$ & $\to$ & () $\ \ |$ \textsf{ true } $|$ \textsf{ false}\\
        &     & $|$ & $i\ \ $ $\ |\ $  $\ "s"$\\
        &     & $|$ & ($x_1,x_2, ..., x_n$)\\
        &     & $|$ & \textsf{in}$i\ x$\\
        &     & $|$ &  $\lambda k\ x.K$ \\
        &     & $|$ &  $\texttt{\#}i\ x$ \\

(primitive & $PrimOp$ & $\to$ & $+\ $ $|\ -\ |$ $\ *$\\
operations) &     & $|$ & $>\ |\ <\ |\ =$\\
        &     & $|$ & $\textsf{andalso }\ |\ \textsf{orelse }\ |\ \textsf{ not}$\\
        &     & $|$ & \textsf{print} $|$ \textsf{int2string}\\
\end{tabular}
  \caption{Defunc syntax}
  \label{fig-sub}
\end{figure}


\section{Defunctionalization}

The functions performing defunctionalization is defined in Figure 5, 6, 7.
The functions translate CPS syntax into Defunc syntax by adding case branches
into the function $applyfunc$ and $applycont$, to represent function definitions.

Function $\textsf{replace}(x, y, K)$ replaces the free existences of $x$ with $y$
in term $K$.

\newgeometry{left=2.5cm,bottom=2cm}

\begin{figure}[!ht]
\begin{tabular}{c}
$K_1, applyfunc(fa, ka, xa)\{K_3\}, applycont(ka', xa')\{K_4\}\leadsto$ \kern2cm\\
         \kern5cm  $ K'_1, applyfunc(fa, ka, xa)\{K_5\}, applycont(ka', xa')\{K_6\}$\\
 $ K_2, applyfunc(fa, ka, xa)\{K_5\}, applycont(ka', xa')\{K_6\} \leadsto$ \kern2cm\\
$$\infer[]{\textsf{letcont }k\ x=K_1\textsf{ in }K_2, applyfunc(fa, ka, xa)\{K_3\},
    applycont(ka', xa')\{K_4\}\leadsto \kern2cm }
        { \kern5cm  K'_2, applyfunc(fa, ka, xa)\{K_7\}, applycont(ka', xa')
            \{\textsf{case}\ ka'\ \textsf{of}\ \vec{b}\} }$$\\\
 \kern3.6cm $\textsf{letval}\ env=(y_1,\ y_2,\ ...\ ,\ y_m)\ \textsf{in}$
    $\textsf{letval}\ k=\texttt{in}i\ env\ \textsf{in}\ K'_2,$\\
   \kern-0.9cm   $applyfunc(fa, ka, xa)\{K_7\},$\\
  \kern4.6cm  $applycont(ka', xa')\{\textsf{case}\ ka'\ \textsf{of}
           \ \textsf{in}i\ env \Rightarrow $
           $ \textsf{letval }y_1=\texttt{\#}1\ env\ \textsf{in}$\\
            \kern11cm   $\textsf{letval }y_2=\texttt{\#}2\ env\ \textsf{in}$\\
            \kern11cm   ...\\
            \kern11.2cm   $\textsf{letval }y_m=\texttt{\#}m\ env\ \textsf{in}$\\
           \kern10.6cm $\textsf{replace}(x, xa', K'_1) $\\
           \kern4.5cm $::\vec{b}\}$\\
   $ \textrm{(where}\ i\ \textrm{is a freshed tag value and}
     \ \{y_1,\ y_2,\ ...\ ,\ y_m\}=\mathcal{F}(K_1)/x)$ \\\\       %letcont
$$\infer[]{k\ x, applyfunc(fa, ka, xa)\{K_1\},
   applycont(ka', xa')\{K_2\}
   \leadsto  \kern4cm }{}\\
 \kern4cm  $applycont(k, x),  applyfunc(fa, ka, xa)\{K_1\},
    applycont(ka', xa')\{K_2\}$\\\\  %cont apply
$$\infer[]{f\ k\ x, applyfunc(fa, ka, xa)\{K_1\},
   applycont(ka', xa')\{K_2\} \leadsto  \kern4cm }{}\\
 \kern4cm  $applyfunc(f, k, x),  applyfunc(fa, ka, xa)\{K_1\},
    applycont(ka', xa')\{K_2\}$\\\\  %func apply
$K_1, applyfunc(fa, ka, xa)\{T\}, applycont(ka', xa')\{T'\} \leadsto $ \kern2cm\\
       \kern5cm $ K_1', applyfunc(fa, ka, xa)\{T_1\}, applycont(ka', xa')\{T'_1\}$\\
$K_2, applyfunc(fa, ka, xa)\{T_1\}, applycont(ka', xa')\{T'_1\} \leadsto $ \kern2cm\\
       \kern5cm $ K_2', applyfunc(fa, ka, xa)\{T_2\}, applycont(ka', xa')\{T'_2\}$\\
       ...\\
$K_m, applyfunc(fa, ka, xa)\{T_{m-1}\}, applycont(ka', xa')\{T'_{m-1}\}
 \leadsto $ \kern2cm\\
       \kern5cm $ K_m', applyfunc(fa, ka, xa)\{T_m\}, applycont(ka', xa')\{T'_m\}$\\
$$\infer[]{\textsf{case}\ x\ \textsf{of}
           \ \overrightarrow{\texttt{in}i_j\ x_j\ \texttt{=>}\ K_j},
          applyfunc(fa, ka, xa)\{T\},
   applycont(ka', xa')\{T'\}
   \leadsto  \kern2cm }
    {}\\
\kern2cm $\textsf{case}\ x\ \textsf{of}
           \ \overrightarrow{\texttt{in}i_j\ x_j\ \texttt{=>}\ K'_j},
          applyfunc(fa, ka, xa)\{T_m\},
   applycont(ka', xa')\{T'_m\}$\\
$ (\textrm{where}\ j=1,\ 2,\ ...\ ,\ m$ )\\\\       %case
$K, applyfunc(fa, ka, xa)\{T_1\}, applycont(ka', xa')\{T_2\} \leadsto $ \kern2cm\\
       \kern5cm $ K_1, applyfunc(fa, ka, xa)\{T_3\}, applycont(ka', xa')\{T_4\}$\\
$$\infer[]{\textsf{letprim }x=PrimOp\ \vec{y}\texttt{ in}\ K,
  applyfunc(fa, ka, xa)\{T_1\}, applycont(ka', xa')\{T_2\}
   \leadsto  \kern2cm }{}\\
\kern2cm  $ \textsf{letprim }x=PrimOp\ \vec{y}\texttt{ in}\ K_1
   ,  applyfunc(fa, ka, xa)\{T_3\},   applycont(ka', xa')\{T_4\}$\\\\   %let prims
$$\infer[]{\textsf{if }x\textsf{ then }k_1\ \textsf{else }k_2 ,
  applyfunc(fa, ka, xa)\{T_1\}, applycont(ka', xa')\{T_2\}
   \leadsto  \kern2cm }{}\\
 \kern3cm $  \textsf{letval }x_1=()\textsf{ in}$
  $  \textsf{if }x\textsf{ then } applycont(k_1\ x_1)
  \textsf{ else } applycont(k_2\ x_1),$\\
 \kern1cm $ applyfunc(fa, ka, xa)\{T_1\}, applycont(ka', xa')\{T_2\}$\\ %if
\end{tabular}
 \caption{Defunctionalization for CPS terms}
  \label{fig-sub}
\end{figure}

\begin{figure}[!ht]
\begin{tabular}{c}
$K_1, applyfunc(fa, ka, xa)\{K_3\}, applycont(ka_1, xa_1)\{K_4\}\leadsto$ \kern2cm\\
         \kern5cm  $ K'_1, applyfunc(fa, ka, xa)\{K_5\}, applycont(ka', xa')\{K_6\}$\\
 $ K_2, applyfunc(fa, ka, xa)\{K_5\}, applycont(ka', xa')\{K_6\} \leadsto$ \kern2cm\\
$$\infer[]{\textsf{letfix }f\ k\ x=K_1\textsf{ in }K_2, applyfunc(fa, ka, xa)\{K_3\},
    applycont(ka_1, xa_1)\{K_4\}\leadsto \kern2cm }
        {\kern5cm  K'_2, applyfunc(fa, ka, xa)\{\textsf{case}\ fa\ \textsf{of}\ \vec{b}\},
         applycont(ka', xa')\{K_7\}       }$$\\
 \kern2.2cm $\textsf{letval}\ env=(y_1,\ y_2,\ ...\ ,\ y_m)\ \textsf{in}$
    $\textsf{letval}\ f=\texttt{in}i\ env\ \textsf{in}\ K'_2,$\\
   \kern3.3cm   $applyfunc(fa, ka, xa)\{\textsf{case}\ fa\ \textsf{of}
            \ \textsf{in}i\ env \Rightarrow $
             $\textsf{letval}\ f=\texttt{in}i\ env\ \textsf{in}$\\
   \kern10.1cm  $ \textsf{letval }y_1=\texttt{\#}1\ env\ \textsf{in}$\\
            \kern10.1cm   $\textsf{letval }y_2=\texttt{\#}2\ env\ \textsf{in}$\\
            \kern8cm   ...\\
            \kern10.3cm   $\textsf{letval }y_m=\texttt{\#}m\ env\ \textsf{in}$\\
           \kern11.8cm $\textsf{replace}(k, ka, \textsf{replace}(x, xa, K'_1)) $\\
           \kern4cm $::\vec{b}\},$\\
  \kern-3cm  $applycont(ka', xa')\{K_7\}$\\
   $ \textrm{(where}\ i\ \textrm{is a freshed tag value and}
     \ \{y_1,\ y_2,\ ...\ ,\ y_m\}=\mathcal{F}(K_1)-\{f,k,x\})$ \\\\       %letfix
$K_1, applyfunc(fa, ka, xa)\{K_3\}, applycont(ka_1, xa_1)\{K_4\}\leadsto$ \kern2cm\\
         \kern5cm  $ K'_1, applyfunc(fa, ka, xa)\{K_5\}, applycont(ka', xa')\{K_6\}$\\
 $ K_2, applyfunc(fa, ka, xa)\{K_5\}, applycont(ka', xa')\{K_6\} \leadsto$ \kern2cm\\
$$\infer[]{\textsf{letval }x=\lambda k\ z.K_1\textsf{ in }K_2
    , applyfunc(fa, ka, xa)\{K_3\}, applycont(ka_1, xa_1)\{K_4\}\leadsto \kern2cm }
        {\kern5cm  K'_2, applyfunc(fa, ka, xa)\{\textsf{case}\ fa\ \textsf{of}\ \vec{b}\},
         applycont(ka', xa')\{K_7\}       }$$\\
 \kern2.2cm $\textsf{letval}\ env=(y_1,\ y_2,\ ...\ ,\ y_m)\ \textsf{in}$
    $\textsf{letval}\ x=\texttt{in}i\ env\ \textsf{in}\ K'_2,$\\
   \kern3.3cm   $applyfunc(fa, ka, xa)\{\textsf{case}\ fa\ \textsf{of}
            \ \textsf{in}i\ env \Rightarrow $
             $ \textsf{letval }y_1=\texttt{\#}1\ env\ \textsf{in}$\\
            \kern10.1cm   $\textsf{letval }y_2=\texttt{\#}2\ env\ \textsf{in}$\\
            \kern8cm   ...\\
            \kern10.3cm   $\textsf{letval }y_m=\texttt{\#}m\ env\ \textsf{in}$\\
           \kern11.8cm $\textsf{replace}(k, ka, \textsf{replace}(z, xa, K'_1)) $\\
           \kern4cm $::\vec{b}\},$\\
  \kern-3cm  $applycont(ka', xa')\{K_7\}$\\
   $ \textrm{(where}\ i\ \textrm{is a freshed tag value and}
     \ \{y_1,\ y_2,\ ...\ ,\ y_m\}=\mathcal{F}(K_1)-\{k,z\})$ \\\\       %letval func
$K, applyfunc(fa, ka, xa)\{T_1\}, applycont(ka', xa')\{T_2\} \leadsto $ \kern2cm\\
       \kern5cm $ K_1, applyfunc(fa, ka, xa)\{T_3\}, applycont(ka', xa')\{T_4\}$\\
$$\infer[]{\textsf{letval }x=V\texttt{ in}\ K,
  applyfunc(fa, ka, xa)\{T_1\}, applycont(ka', xa')\{T_2\}
   \leadsto  \kern2cm }{}\\
\kern3cm  $ \textsf{letval }x=V\texttt{ in}\ K_1
   ,  applyfunc(fa, ka, xa)\{T_3\},   applycont(ka', xa')\{T_4\}$\\
(given that $V$ is not a lambda abstraction.)   %letval
\end{tabular}
 \caption{Defunctionalization for CPS terms(continued)}
  \label{fig-sub}
\end{figure}

\restoregeometry

\section{Code generation}

The code generation process for the defunctionalized syntax tree is intuitive.
One thing to be noticed is that the \textsf{case} expression may contain a lot of
bindings (in contrast, if we perform code generation following the closure conversion
process, only a tuple may be defined there in case of function calls).
In this way, the C code for each switch case should be in a block (surrounded by
a pair of \texttt{"\{\}"}).

Moreover, as the $appl\_func$ function and $apply\_cont$ function may be
 mutually recursive, we may need to output the function declarations for them first.

The modifications should also be included when generation code for garbage collection.\\\\

\noindent{
Updates:
\begin{description}
   \item[15-9-5:]
  Added boolean values and corresponding operations. Changed \texttt{if0}
  expression into \texttt{if} expression.
\end{description}
}

\end{document}
